\documentclass[journal,12pt,twocolumn]{IEEEtran}
%
\usepackage{setspace}
\usepackage{gensymb}
\usepackage{siunitx}
\usepackage{tkz-euclide} 
\usepackage{textcomp}
\usepackage{standalone}
\usetikzlibrary{calc}
\newcommand\hmmax{0}
\newcommand\bmmax{0}

%\doublespacing
\singlespacing

%\usepackage{graphicx}
%\usepackage{amssymb}
%\usepackage{relsize}
\usepackage[cmex10]{amsmath}
%\usepackage{amsthm}
%\interdisplaylinepenalty=2500
%\savesymbol{iint}
%\usepackage{txfonts}
%\restoresymbol{TXF}{iint}
%\usepackage{wasysym}
\usepackage{amsthm}
%\usepackage{iithtlc}
\usepackage{mathrsfs}
\usepackage{txfonts}
\usepackage{stfloats}
\usepackage{bm}
\usepackage{cite}
\usepackage{cases}
\usepackage{subfig}
%\usepackage{xtab}
\usepackage{longtable}
\usepackage{multirow}
%\usepackage{algorithm}
%\usepackage{algpseudocode}
\usepackage{enumitem}
\usepackage{mathtools}
\usepackage{steinmetz}
\usepackage{tikz}
\usepackage{circuitikz}
\usepackage{verbatim}
\usepackage{tfrupee}
\usepackage[breaklinks=true]{hyperref}
%\usepackage{stmaryrd}
\usepackage{tkz-euclide} % loads  TikZ and tkz-base
%\usetkzobj{all}
\usetikzlibrary{calc,math}
\usepackage{listings}
    \usepackage{color}                                            %%
    \usepackage{array}                                            %%
    \usepackage{longtable}                                        %%
    \usepackage{calc}                                             %%
    \usepackage{multirow}                                         %%
    \usepackage{hhline}                                           %%
    \usepackage{ifthen}                                           %%
  %optionally (for landscape tables embedded in another document): %%
    \usepackage{lscape}     
\usepackage{multicol}
\usepackage{chngcntr}
\usepackage{amsmath}
\usepackage{cleveref}
%\usepackage{enumerate}

%\usepackage{wasysym}
%\newcounter{MYtempeqncnt}
\DeclareMathOperator*{\Res}{Res}
%\renewcommand{\baselinestretch}{2}
\renewcommand\thesection{\arabic{section}}
\renewcommand\thesubsection{\thesection.\arabic{subsection}}
\renewcommand\thesubsubsection{\thesubsection.\arabic{subsubsection}}

\renewcommand\thesectiondis{\arabic{section}}
\renewcommand\thesubsectiondis{\thesectiondis.\arabic{subsection}}
\renewcommand\thesubsubsectiondis{\thesubsectiondis.\arabic{subsubsection}}

% correct bad hyphenation here
\hyphenation{op-tical net-works semi-conduc-tor}
\def\inputGnumericTable{}                                 %%

\lstset{
%language=C,
frame=single, 
breaklines=true,
columns=fullflexible
}
%\lstset{
%language=tex,
%frame=single, 
%breaklines=true
%}
\usepackage{graphicx}
\usepackage{pgfplots}

\begin{document}


\newtheorem{theorem}{Theorem}[section]
\newtheorem{problem}{Problem}
\newtheorem{proposition}{Proposition}[section]
\newtheorem{lemma}{Lemma}[section]
\newtheorem{corollary}[theorem]{Corollary}
\newtheorem{example}{Example}[section]
\newtheorem{definition}[problem]{Definition}
%\newtheorem{thm}{Theorem}[section] 
%\newtheorem{defn}[thm]{Definition}
%\newtheorem{algorithm}{Algorithm}[section]
%\newtheorem{cor}{Corollary}
\newcommand{\BEQA}{\begin{eqnarray}}
\newcommand{\EEQA}{\end{eqnarray}}
\newcommand{\define}{\stackrel{\triangle}{=}}
\bibliographystyle{IEEEtran}
%\bibliographystyle{ieeetr}
\providecommand{\mbf}{\mathbf}
\providecommand{\abs}[1]{\ensuremath{\left\vert#1\right\vert}}
\providecommand{\norm}[1]{\ensuremath{\left\lVert#1\right\rVert}}
\providecommand{\mean}[1]{\ensuremath{E\left[ #1 \right]}}
\providecommand{\pr}[1]{\ensuremath{\Pr\left(#1\right)}}
\providecommand{\qfunc}[1]{\ensuremath{Q\left(#1\right)}}
\providecommand{\sbrak}[1]{\ensuremath{{}\left[#1\right]}}
\providecommand{\lsbrak}[1]{\ensuremath{{}\left[#1\right.}}
\providecommand{\rsbrak}[1]{\ensuremath{{}\left.#1\right]}}
\providecommand{\brak}[1]{\ensuremath{\left(#1\right)}}
\providecommand{\lbrak}[1]{\ensuremath{\left(#1\right.}}
\providecommand{\rbrak}[1]{\ensuremath{\left.#1\right)}}
\providecommand{\cbrak}[1]{\ensuremath{\left\{#1\right\}}}
\providecommand{\lcbrak}[1]{\ensuremath{\left\{#1\right.}}
\providecommand{\rcbrak}[1]{\ensuremath{\left.#1\right\}}}
\theoremstyle{remark}
\newtheorem{rem}{Remark}
\newcommand{\sgn}{\mathop{\mathrm{sgn}}}
\providecommand{\res}[1]{\Res\displaylimits_{#1}} 
%\providecommand{\norm}[1]{\lVert#1\rVert}
\providecommand{\mtx}[1]{\mathbf{#1}}
\providecommand{\fourier}{\overset{\mathcal{F}}{ \rightleftharpoons}}
%\providecommand{\hilbert}{\overset{\mathcal{H}}{ \rightleftharpoons}}
\providecommand{\system}{\overset{\mathcal{H}}{ \longleftrightarrow}}
	%\newcommand{\solution}[2]{\textbf{Solution:}{#1}}
\newcommand{\solution}{\noindent \textbf{Solution: }}
\newcommand{\cosec}{\,\text{cosec}\,}
\providecommand{\dec}[2]{\ensuremath{\overset{#1}{\underset{#2}{\gtrless}}}}
\newcommand{\myvec}[1]{\ensuremath{\begin{pmatrix}#1\end{pmatrix}}}
\newcommand{\mydet}[1]{\ensuremath{\begin{vmatrix}#1\end{vmatrix}}}
%\numberwithin{equation}{section}
\numberwithin{equation}{subsection}
%\numberwithin{problem}{section}
%\numberwithin{definition}{section}
\makeatletter
\@addtoreset{figure}{problem}
\makeatother
\let\StandardTheFigure\thefigure
\let\vec\mathbf
%\renewcommand{\thefigure}{\theproblem.\arabic{figure}}
\renewcommand{\thefigure}{\theproblem}
%\setlist[enumerate,1]{before=\renewcommand\theequation{\theenumi.\arabic{equation}}
%\counterwithin{equation}{enumi}
%\renewcommand{\theequation}{\arabic{subsection}.\arabic{equation}}
\def\putbox#1#2#3{\makebox[0in][l]{\makebox[#1][l]{}\raisebox{\baselineskip}[0in][0in]{\raisebox{#2}[0in][0in]{#3}}}}
     \def\rightbox#1{\makebox[0in][r]{#1}}
     \def\centbox#1{\makebox[0in]{#1}}
     \def\topbox#1{\raisebox{-\baselineskip}[0in][0in]{#1}}
     
 \vspace{3cm}
 \title{Assignment 17}
 \author{Matish Singh Tanwar}
 \maketitle
 \newpage
 \bigskip
 %\renewcommand{\thefigure}{\theenumi}
 \renewcommand{\thetable}{\theenumi}
\vspace{1.0cm}
\begin{abstract}
This document solves a problem of Linear Algebra.
\end{abstract}
\vspace{0.5cm}
%
Download all latex-tikz codes from 
\begin{lstlisting}
https://github.com/Matish007/Matrix-Theory-EE5609-/tree/master/Assignment_17
\end{lstlisting}
%
%
\vspace{0.5mm}
\section{Problem}
Let $\vec{P}$ be a $2\times 2$ complex matrix such that
\begin{align}
\vec{P}^\theta \vec{P} =\vec{I}
\end{align}
where $\vec{P}^\theta$ is the conjugate transpose of $\vec{P}$.Then the eigen values of $\vec{P}$ are
\begin{enumerate}
\item{real}
\item{complex conjugates of each othe}
\item{reciprocals of each other}
\item{of modulus 1}
\end{enumerate}
\section{SOLUTION}
 \renewcommand{\thetable}{1}
\begin{table*}[ht!]
\begin{center}
\begin{tabular}{|c|c|}
\hline
\textbf{Options} & \textbf{Explanation} \\
\hline
\text{REAL} & \\
Counter Example & $\vec{P}=\myvec{i&0\\0&i}$\\
&\\
&$\vec{P}^\theta=\myvec{-i&0\\0&-i}$\\
&\\
&$\vec{P}^\theta\vec{P}=\myvec{1&0\\0&1}=\vec{I}$\\
& Eigen values of $\vec{P}$ are $i,i$ which are not real\\
& Hence,incorrect.\\
\hline
\text{Complex Conjugates of each other.} & 
From above,$(i,i)$ are not complex conjugate\\& of each other\\
&\\
&Hence,incorrect.\\
\hline
\text{Reciprocals of each other}
& Reciprocal of $i=\frac{1}{i}=\frac{i^4}{i}=i^3\neq i$\\
&\\
& Hence,incorrect.\\
\hline

\text{of modulus 1}&\\
Proof&$\vec{P}\vec{V}=\lambda\vec{V}$\\
&where,$\vec{V}$ is eigen vector of $\vec{P}$ and \\&$\lambda$ is eigen value of $\vec{P}$ \\
&\\
&Taking conjugate transpose on both sides,we get\\
&$\vec{V}^\theta \vec{P}^\theta=\lambda^\theta\vec{V}^\theta$\\
&\\
&$\vec{V}^\theta \vec{P}^\theta\vec{P}\vec{V}=\lambda^\theta\vec{V}^\theta\lambda\vec{V}\qquad,\because \vec{P}\vec{V}=\lambda\vec{V}$\\
&\\
&$\vec{V}^\theta \vec{I}\vec{V}=\lambda^\theta\lambda\vec{V}^\theta\vec{V}\qquad,\because \vec{P^\theta}\vec{P}=\vec{I}$\\
&\\
&$(1-\lambda^\theta\lambda)\vec{V}^\theta\vec{V}=0$\\
&Since,$\vec{V}$ is not zero.\\
&$(1-\lambda^\theta\lambda)=0$\\
&$\lambda^\theta\lambda=1$\\
&\\
&$\norm{\lambda}^2=1$\\
&\\
&$\lambda=1$\\
&\\
&Hence,correct.\\
\hline
\end{tabular}
\caption{Finding Correct Option}
\label{table1}
\end{center}
\end{table*}
 
\end{document}
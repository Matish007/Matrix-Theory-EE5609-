\documentclass[journal,12pt,twocolumn]{IEEEtran}

\usepackage{setspace}
\usepackage{gensymb}

\singlespacing


\usepackage[cmex10]{amsmath}

\usepackage{amsthm}

\usepackage{mathrsfs}
\usepackage{txfonts}
\usepackage{stfloats}
\usepackage{bm}
\usepackage{cite}
\usepackage{cases}
\usepackage{subfig}

\usepackage{longtable}
\usepackage{multirow}

\usepackage{enumitem}
\usepackage{mathtools}
%\usepackage{steinmetz}
\usepackage{tikz}
\usepackage{circuitikz}
\usepackage{verbatim}
%\usepackage{tfrupee}
\usepackage[breaklinks=true]{hyperref}

\usepackage{tkz-euclide}

\usetikzlibrary{calc,math}
\usepackage{listings}
    \usepackage{color}                                            %%
    \usepackage{array}                                            %%
    \usepackage{longtable}                                        %%
    \usepackage{calc}                                             %%
    \usepackage{multirow}                                         %%
    \usepackage{hhline}                                           %%
    \usepackage{ifthen}                                           %%
    \usepackage{lscape}     
\usepackage{multicol}
\usepackage{chngcntr}

\DeclareMathOperator*{\Res}{Res}

\renewcommand\thesection{\arabic{section}}
\renewcommand\thesubsection{\thesection.\arabic{subsection}}
\renewcommand\thesubsubsection{\thesubsection.\arabic{subsubsection}}

\renewcommand\thesectiondis{\arabic{section}}
\renewcommand\thesubsectiondis{\thesectiondis.\arabic{subsection}}
\renewcommand\thesubsubsectiondis{\thesubsectiondis.\arabic{subsubsection}}


\hyphenation{op-tical net-works semi-conduc-tor}
\def\inputGnumericTable{}                                 %%

\lstset{
%language=C,
frame=single, 
breaklines=true,
columns=fullflexible
}
\begin{document}
\begin{center}
\huge Fibonacci Question\\

\large Matish Singh Tanwar\\
\large AI20MTECH11005\\
\end{center}
\vspace{1.0cm}
\begin{abstract}
This document finds out the fibonacci number in O(logn) time.
\end{abstract}
\vspace{0.5cm}
Download all latex codes from
\begin{lstlisting}
https://github.com/Matish007/Matrix-Theory-EE5609-/tree/master/Fibonacci
\end{lstlisting}
%
\vspace{0.5mm}
\section{Problem}
Find the fibonacci number using Matrix in O(logn) time.
\section{Explanation}
First thing which come in our mind is the equation of fibonacci number i.e.
\begin{align}
\boxed{f_n=f_n-1 + f_n-2}.
\end{align}
Now,how to write this whole equation in matrix form.Let's try
\begin{align}
    \begin{pmatrix}f_n\end{pmatrix}=\begin{pmatrix}1 & 1\end{pmatrix}\begin{pmatrix}f_n-1\\f_n-2\end{pmatrix}
\end{align}
If we solve the equation (2) we will get the equation (1).\\
For finding $f_n$ we require two values $f_n-1$ and $f_n-2$ again we have to calculate for them if further values are not known to us then again calculate them.So,equation (2) is just the matrix form of equation(1).But we cannot calculate the $n^{th}$ fibonacci number in less than O(n) time using equation (2) because we cannot self multiply the matrix $\begin{pmatrix}1 & 1\end{pmatrix}$ due to its order.We will modify our equation (2) for finding number in less computation time.But some important things can be inferred out from equation (2) which can be used for determining a matrix which can find $n^{th}$ fibonacci number in less time which are as follows:-
\begin{itemize}
    \item We need two values $f_n-1$,$f_n-2$ for calculation of $f_n$
    \item We will first try for 2X2 matrix as then self multiplication would be possible.
    \item If we see equation (2) then first row of matrix $\begin{pmatrix}1 & 1\end{pmatrix}$ and first column of matrix $\begin{pmatrix} f_n-1 & f_n-2\end{pmatrix}^T$ is giving us $f_n$.
    \item So,our resultant 2X2 matrix should have non zero first column and first row.
    \item Now L.H.S in equation(2) should be of order 2X1.
    So,we should add one more entry at L.H.S of equation (2).\\
    \item L.H.S contain $f_n$ so we should create a entry of $f_n-1$.
    \item Now,we should change our L.H.S matrix of R.H.S of equation (2) according to L.H.S
    \item How our equation is now?
    \begin{align}
      \begin{pmatrix}f_n\\f_n-1\end{pmatrix}=\begin{pmatrix}1 & 1\end{pmatrix}\begin{pmatrix}f_n-1\\f_n-2\end{pmatrix}
    \end{align}
    \item We should add such row in our matrix such that it is equal to L.H.S of equation (3).
    \item 1 0 should be added so our matrix becomes\\
    \begin{align}
    \begin{pmatrix}1 & 1\\1 & 0\end{pmatrix}
    \end{align}
    \item Replacing (4) in (3) we get
    \begin{align}
      \boxed{\begin{pmatrix}f_n\\f_n-1\end{pmatrix}=\begin{pmatrix}1 & 1\\1 & 0\end{pmatrix}\begin{pmatrix}f_n-1\\f_n-2\end{pmatrix}}
    \end{align}
    
\end{itemize}
Since equation (4) matrix is of order 2X2,we can perform multiplication of itself contradictory to equation (2) matrix whose order was 1X2.We cannot multiply that matriix to itself that's why we have to come through this whole process.\\
If initial values are $f_0$ and $f_1$,We can rewrite equation 5 as:-
\begin{align}
      \boxed{\begin{pmatrix}f_n\\f_n-1\end{pmatrix}=\begin{pmatrix}1 & 1\\1 & 0\end{pmatrix}^{n-1}\begin{pmatrix}f_n-1\\f_n-2\end{pmatrix}}
      \end{align}
Now,you are thinking about we still have to do matrix multiplication n times,So it is taking O(n).\\
Well we can calculate $\mathbf{A^{n}}$ in O(logn) time.Here,$\mathbf{A}$ is a matrix.
We can do this by using divide and conquer technique like this:-
\begin{itemize}
    \item Let n=$2^m$
    \item Now divide n by two
    \item Combine them by using recurrence.
    \item Now conquer them by multiplying to itself two times
\end{itemize}
Let's understand by a example:-
Suppose n=8,$2^m$=8 $\implies m=3$\\ and we have to calculate $\mathbf{A^8}$\\
8/2=4\\
4/2=2\\
2/2=1=$2^0$\\
$\mathbf{A^2}$=$\mathbf{A}$.$\mathbf{A}$\\
$\mathbf{A^4}$=$\mathbf{A^2}$.$\mathbf{A^2}$\\
$\mathbf{A^8}$=$\mathbf{A^4}$.$\mathbf{A^4}$\\
For $\mathbf{A^8}$ literally we required 3 multiplications only and we got $\mathbf{A^2}$.
So,we just required O(logn) time to calculate $\mathbf{A^n}$.

Hence,we can find $n^{th}$ fibonacci number in just O(logn) time.

\end{document}

\documentclass[journal,12pt,twocolumn]{IEEEtran}

\usepackage{setspace}
\usepackage{gensymb}

\singlespacing


\usepackage[cmex10]{amsmath}
\usepackage{amsthm}
\usepackage{mathrsfs}
\usepackage{txfonts}
\usepackage{stfloats}
\usepackage{bm}
\usepackage{cite}
\usepackage{cases}
\usepackage{subfig}
\usepackage{longtable}
\usepackage{multirow}
\usepackage{mathtools}
\usepackage{steinmetz}
\usepackage{tikz}
\usepackage{circuitikz}
\usepackage{verbatim}
\usepackage{tfrupee}
\usepackage[breaklinks=true]{hyperref}
\usepackage{tkz-euclide} % loads  TikZ and tkz-base
%\usetkzobj{all}
\usetikzlibrary{calc,math}
\usepackage{listings}
    \usepackage{color}                                            %%
    \usepackage{array}                                            %%
    \usepackage{longtable}                                        %%
    \usepackage{calc}                                             %%
    \usepackage{multirow}                                         %%
    \usepackage{hhline}                                           %%
    \usepackage{ifthen}                                           %%
  %optionally (for landscape tables embedded in another document): %%
    \usepackage{lscape}     
\usepackage{multicol}
\usepackage{chngcntr}
\DeclareMathOperator*{\Res}{Res}
\renewcommand\thesection{\arabic{section}}
\renewcommand\thesubsection{\thesection.\arabic{subsection}}
\renewcommand\thesubsubsection{\thesubsection.\arabic{subsubsection}}

\renewcommand\thesectiondis{\arabic{section}}
\renewcommand\thesubsectiondis{\thesectiondis.\arabic{subsection}}
\renewcommand\thesubsubsectiondis{\thesubsectiondis.\arabic{subsubsection}}

\DeclarePairedDelimiter\abs{\lvert}{\rvert} % \abs{} for numerisk værdi
\DeclarePairedDelimiter\norm{\lVert}{\rVert}
\makeatletter
\let\oldabs\abs
\def\abs{\@ifstar{\oldabs}{\oldabs*}}
\let\oldnorm\norm
\def\norm{\@ifstar{\oldnorm}{\oldnorm*}}
\makeatother


\newcommand{\bignorm}[1]{\Bigl \| #1 \Bigr \| #1}
% correct bad hyphenation here
\hyphenation{op-tical net-works semi-conduc-tor}
\def\inputGnumericTable{}                                 %%

\lstset{
frame=single, 
breaklines=true,
columns=fullflexible
}


\begin{document}
\begin{center}
\huge Question\\

\large Matish Singh Tanwar\\
\large AI20MTECH11005\\
\end{center}
\vspace{1.0cm}
\begin{abstract}
This document solves the system of linear equations using Gaussian Elimination\\
\end{abstract}
\vspace{0.5cm}
Download all latex codes from 
\begin{lstlisting}
https://github.com/Matish007/Matrix-Theory-EE5609-/tree/master/Question
\end{lstlisting}
%
\vspace{0.5mm}
\section{Problem}
Find all binary solutions using Gaussian Elimination of:-
\begin{align}
    \mathbf{x}\oplus\mathbf{z}=1\\
     \mathbf{x}\oplus\mathbf{y}\oplus\mathbf{z}=1\\
      \mathbf{y}\oplus\mathbf{z}=0
\end{align}
\section{Explanation}
According to Gauss Elimination we will apply operations on coefficient matrix $\mathbf{A}$ such that it is converted to Upper triangular matrix $\mathbf{U}$.It will be a success if we get all 3 pivots otherwise failure.If success,then by Back Substitution,we will find the values of $\mathbf{x}$,$\mathbf{y}$,$\mathbf{z}$. We will merge $\mathbf{b}$ also in $\mathbf{A}$ , so it will become a Augmented Matrix. 
\begin{align}
 \mathbf{A}=\begin{pmatrix}
1 & 0 & 1\\
1 & 1 & 1\\
0 & 1 & 1
\end{pmatrix}\\
\mathbf{b}=\begin{pmatrix}
1\\
1\\
0
\end{pmatrix}\\
\mathbf{A}|\mathbf{b}=\left(\begin{array}{ccc|c}  
 1 & 0 & 1 & 1\\  
 1 & 1 & 1 & 1\\
 0 & 1 & 1 & 0
\end{array}\right)
\end{align}
where equation (4) is our coefficient matrix and equation (6) is our Augmented Matrix\\
Now,applying operations on the augmented matrix:-
\begin{align}
   \left(\begin{array}{ccc|c}  
 1 & 0 & 1 & 1\\  
 1 & 1 & 1 & 1\\
 0 & 1 & 1 & 0
\end{array}\right)\xleftrightarrow[]{R2\leftarrow R2-R1} \left(\begin{array}{ccc|c}  
 1 & 0 & 1 & 1\\  
 0 & 1 & 0 & 0\\
 0 & 1 & 1 & 0
\end{array}\right)\xleftrightarrow[]{R3\leftarrow R3-R2}\left(\begin{array}{ccc|c}  
 1 & 0 & 1 & 1\\  
 0 & 1 & 0 & 0\\
 0 & 0 & 1 & 0
\end{array}\right)
\end{align}
Equation (7) final result is $\mathbf{U}|\mathbf{c}$
Where $\mathbf{U}$ is Upper triangular Matrix,Coefficient Matrix $\mathbf{A}$ got converted to $\mathbf{U}$ and $\mathbf{b}$ got converted to $\mathbf{c}$.
We converted (1),(2),(3) equations in a simplified way which is represented by $\mathbf{U}|\mathbf{c}$ matrix.Let's see what we got\\
\begin{align}
    \mathbf{x}\oplus\mathbf{z}=1\\
     \mathbf{y}=0\\
      \mathbf{z}=0
\end{align}\\
From equation (9) and (10) it is clear that $\mathbf{y}$=0 and $\mathbf{z}$=0\\
Substituting (10) in (8) we get \\
\begin{align}
    \mathbf{x}=1
\end{align}\\

Equation (9),(10),(11) combinely is our required solution of the linear system of equations. 

\end{document}
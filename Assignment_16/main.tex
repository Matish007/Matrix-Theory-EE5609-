\documentclass[journal,12pt,twocolumn]{IEEEtran}
%
\usepackage{setspace}
\usepackage{gensymb}
\usepackage{siunitx}
\usepackage{tkz-euclide} 
\usepackage{textcomp}
\usepackage{standalone}
\usetikzlibrary{calc}
\newcommand\hmmax{0}
\newcommand\bmmax{0}

%\doublespacing
\singlespacing

%\usepackage{graphicx}
%\usepackage{amssymb}
%\usepackage{relsize}
\usepackage[cmex10]{amsmath}
%\usepackage{amsthm}
%\interdisplaylinepenalty=2500
%\savesymbol{iint}
%\usepackage{txfonts}
%\restoresymbol{TXF}{iint}
%\usepackage{wasysym}
\usepackage{amsthm}
%\usepackage{iithtlc}
\usepackage{mathrsfs}
\usepackage{txfonts}
\usepackage{stfloats}
\usepackage{bm}
\usepackage{cite}
\usepackage{cases}
\usepackage{subfig}
%\usepackage{xtab}
\usepackage{longtable}
\usepackage{multirow}
%\usepackage{algorithm}
%\usepackage{algpseudocode}
\usepackage{enumitem}
\usepackage{mathtools}
\usepackage{steinmetz}
\usepackage{tikz}
\usepackage{circuitikz}
\usepackage{verbatim}
\usepackage{tfrupee}
\usepackage[breaklinks=true]{hyperref}
%\usepackage{stmaryrd}
\usepackage{tkz-euclide} % loads  TikZ and tkz-base
%\usetkzobj{all}
\usetikzlibrary{calc,math}
\usepackage{listings}
    \usepackage{color}                                            %%
    \usepackage{array}                                            %%
    \usepackage{longtable}                                        %%
    \usepackage{calc}                                             %%
    \usepackage{multirow}                                         %%
    \usepackage{hhline}                                           %%
    \usepackage{ifthen}                                           %%
  %optionally (for landscape tables embedded in another document): %%
    \usepackage{lscape}     
\usepackage{multicol}
\usepackage{chngcntr}
\usepackage{amsmath}
\usepackage{cleveref}
%\usepackage{enumerate}

%\usepackage{wasysym}
%\newcounter{MYtempeqncnt}
\DeclareMathOperator*{\Res}{Res}
%\renewcommand{\baselinestretch}{2}
\renewcommand\thesection{\arabic{section}}
\renewcommand\thesubsection{\thesection.\arabic{subsection}}
\renewcommand\thesubsubsection{\thesubsection.\arabic{subsubsection}}

\renewcommand\thesectiondis{\arabic{section}}
\renewcommand\thesubsectiondis{\thesectiondis.\arabic{subsection}}
\renewcommand\thesubsubsectiondis{\thesubsectiondis.\arabic{subsubsection}}

% correct bad hyphenation here
\hyphenation{op-tical net-works semi-conduc-tor}
\def\inputGnumericTable{}                                 %%

\lstset{
%language=C,
frame=single, 
breaklines=true,
columns=fullflexible
}
%\lstset{
%language=tex,
%frame=single, 
%breaklines=true
%}
\usepackage{graphicx}
\usepackage{pgfplots}

\begin{document}


\newtheorem{theorem}{Theorem}[section]
\newtheorem{problem}{Problem}
\newtheorem{proposition}{Proposition}[section]
\newtheorem{lemma}{Lemma}[section]
\newtheorem{corollary}[theorem]{Corollary}
\newtheorem{example}{Example}[section]
\newtheorem{definition}[problem]{Definition}
%\newtheorem{thm}{Theorem}[section] 
%\newtheorem{defn}[thm]{Definition}
%\newtheorem{algorithm}{Algorithm}[section]
%\newtheorem{cor}{Corollary}
\newcommand{\BEQA}{\begin{eqnarray}}
\newcommand{\EEQA}{\end{eqnarray}}
\newcommand{\define}{\stackrel{\triangle}{=}}
\bibliographystyle{IEEEtran}
%\bibliographystyle{ieeetr}
\providecommand{\mbf}{\mathbf}
\providecommand{\abs}[1]{\ensuremath{\left\vert#1\right\vert}}
\providecommand{\norm}[1]{\ensuremath{\left\lVert#1\right\rVert}}
\providecommand{\mean}[1]{\ensuremath{E\left[ #1 \right]}}
\providecommand{\pr}[1]{\ensuremath{\Pr\left(#1\right)}}
\providecommand{\qfunc}[1]{\ensuremath{Q\left(#1\right)}}
\providecommand{\sbrak}[1]{\ensuremath{{}\left[#1\right]}}
\providecommand{\lsbrak}[1]{\ensuremath{{}\left[#1\right.}}
\providecommand{\rsbrak}[1]{\ensuremath{{}\left.#1\right]}}
\providecommand{\brak}[1]{\ensuremath{\left(#1\right)}}
\providecommand{\lbrak}[1]{\ensuremath{\left(#1\right.}}
\providecommand{\rbrak}[1]{\ensuremath{\left.#1\right)}}
\providecommand{\cbrak}[1]{\ensuremath{\left\{#1\right\}}}
\providecommand{\lcbrak}[1]{\ensuremath{\left\{#1\right.}}
\providecommand{\rcbrak}[1]{\ensuremath{\left.#1\right\}}}
\theoremstyle{remark}
\newtheorem{rem}{Remark}
\newcommand{\sgn}{\mathop{\mathrm{sgn}}}
\providecommand{\res}[1]{\Res\displaylimits_{#1}} 
%\providecommand{\norm}[1]{\lVert#1\rVert}
\providecommand{\mtx}[1]{\mathbf{#1}}
\providecommand{\fourier}{\overset{\mathcal{F}}{ \rightleftharpoons}}
%\providecommand{\hilbert}{\overset{\mathcal{H}}{ \rightleftharpoons}}
\providecommand{\system}{\overset{\mathcal{H}}{ \longleftrightarrow}}
	%\newcommand{\solution}[2]{\textbf{Solution:}{#1}}
\newcommand{\solution}{\noindent \textbf{Solution: }}
\newcommand{\cosec}{\,\text{cosec}\,}
\providecommand{\dec}[2]{\ensuremath{\overset{#1}{\underset{#2}{\gtrless}}}}
\newcommand{\myvec}[1]{\ensuremath{\begin{pmatrix}#1\end{pmatrix}}}
\newcommand{\mydet}[1]{\ensuremath{\begin{vmatrix}#1\end{vmatrix}}}
%\numberwithin{equation}{section}
\numberwithin{equation}{subsection}
%\numberwithin{problem}{section}
%\numberwithin{definition}{section}
\makeatletter
\@addtoreset{figure}{problem}
\makeatother
\let\StandardTheFigure\thefigure
\let\vec\mathbf
%\renewcommand{\thefigure}{\theproblem.\arabic{figure}}
\renewcommand{\thefigure}{\theproblem}
%\setlist[enumerate,1]{before=\renewcommand\theequation{\theenumi.\arabic{equation}}
%\counterwithin{equation}{enumi}
%\renewcommand{\theequation}{\arabic{subsection}.\arabic{equation}}
\def\putbox#1#2#3{\makebox[0in][l]{\makebox[#1][l]{}\raisebox{\baselineskip}[0in][0in]{\raisebox{#2}[0in][0in]{#3}}}}
     \def\rightbox#1{\makebox[0in][r]{#1}}
     \def\centbox#1{\makebox[0in]{#1}}
     \def\topbox#1{\raisebox{-\baselineskip}[0in][0in]{#1}}
     
 \vspace{3cm}
 \title{Assignment 16}
 \author{Matish Singh Tanwar}
 \maketitle
 \newpage
 \bigskip
 %\renewcommand{\thefigure}{\theenumi}
 \renewcommand{\thetable}{\theenumi}
\vspace{1.0cm}
\begin{abstract}
This document solves a problem of Linear Algebra.
\end{abstract}
\vspace{0.5cm}
%
Download all latex-tikz codes from 
\begin{lstlisting}
https://github.com/Matish007/Matrix-Theory-EE5609-/tree/master/Assignment_16
\end{lstlisting}
%
%
\vspace{0.5mm}
\section{Problem}
Let
\begin{align}
    \vec{A}=\myvec{x&y\\-y&x}\label{1}
\end{align}
where x,y $\in \mathbb{R}$ such that
\begin{align}
    x^{2}+y^{2}=1
\end{align}
Then,we must have:
\begin{enumerate}
\item{$\vec{A^{n}}=\myvec{\cos\theta& \sin\theta\\-\sin\theta& \cos\theta}$$\forall n\geq{1}$\\ where x=$\cos(\frac{\theta}{n})$,y=$\sin(\frac{\theta}{n})$}
\item{$trace(\vec{A})\neq 0$}
\item{$\vec{A^{T}}=\vec{A^{-1}}$}
\item{$\vec{A}$ is similar to a diagonal matrix over $\mathbb{C}$}
\end{enumerate}
\section{SOLUTION}

 \renewcommand{\thetable}{1}
\begin{table*}[ht!]
\begin{center}
\begin{tabular}{|c|c|}
\hline
\textbf{Options} & \textbf{Explanation} \\
\hline
\text{$\vec{A^{n}}=\myvec{\cos\theta& \sin\theta\\-\sin\theta& \cos\theta}$$\forall n\geq{1}$} & $\vec{A}=\myvec{x&y\\-y&x}$\\
where x=$\cos(\frac{\theta}{n})$,y=$\sin(\frac{\theta}{n})$& 
\\&$\vec{A}=\myvec{\cos(\frac{\theta}{n})&\sin(\frac{\theta}{n})\\-\sin(\frac{\theta}{n})&\cos(\frac{\theta}{n})}$\\
&\\& $\vec{A^2}=\vec{A}.\vec{A} =\myvec{\cos(\frac{\theta}{n})&\sin(\frac{\theta}{n})\\-\sin(\frac{\theta}{n})&\cos(\frac{\theta}{n})}\myvec{\cos(\frac{\theta}{n})&\sin(\frac{\theta}{n})\\-\sin(\frac{\theta}{n})&\cos(\frac{\theta}{n})}$\\
&\\&$\vec{A^2}=\myvec{\cos(\frac{2\theta}{n})&\sin(\frac{2\theta}{n})\\-\sin(\frac{2\theta}{n})&\cos(\frac{2\theta}{n})}$\\
&\\
&$\vec{A^3}=\vec{A^2}.\vec{A}=\myvec{\cos(\frac{2\theta}{n})&\sin(\frac{2\theta}{n})\\-\sin(\frac{2\theta}{n})&\cos(\frac{2\theta}{n})}\myvec{\cos(\frac{\theta}{n})&\sin(\frac{\theta}{n})\\-\sin(\frac{\theta}{n})&\cos(\frac{\theta}{n})}$\\
&\\
& $\vec{A^3}=\myvec{\cos(\frac{3\theta}{n})&\sin(\frac{3\theta}{n})\\-\sin(\frac{3\theta}{n})&\cos(\frac{3\theta}{n})}$\\
&..\\
&..\\
&..\\
&\\
&$\vec{A^{n}}=\myvec{\cos(\frac{n\theta}{n})& \sin(\frac{n\theta}{n})\\-\sin(\frac{n\theta}{n})& \cos(\frac{n\theta}{n})}$\\
&\\& $\vec{A^{n}}=\myvec{\cos\theta& \sin\theta\\-\sin\theta& \cos\theta}$ $\qquad \forall n\geq{1}$\\
& Hence,correct\\
\hline
\text{$trace(\vec{A})\neq 0$} & 
Let,$x=0,y=1$,Substitute in $\eqref{1}$\\
& $\vec{A}=\myvec{0&1\\-1&0}$\\
& $trace(\vec{A})= 0$\\
& Hence,incorrect\\
\hline
\text{$\vec{A^{T}}=\vec{A^{-1}}$}
&$\vec{A}=\myvec{x&y\\-y&x}$\\
&$\vec{A^{T}}=\myvec{x&-y\\y&x}$\\
$\vec{A}\vec{A^{T}}$&$\myvec{x&y\\-y&x}$$\myvec{x&-y\\y&x}$\\
& $\myvec{x^2+y^2&-xy+xy\\-xy+xy&x^2+y^2}$\\
& $\myvec{1&0\\0&1}$\\
& $\vec{A}{A^T}=\vec{I}=\vec{A^T}{A}$\\
& $\implies \vec{A}=\vec{A^{-1}}$\\
& $\implies \vec{A}$ is an orthogonal matrix.\\ 
& Hence,correct.\\
\hline
\end{tabular}
\end{center}
\end{table*}
\renewcommand{\thetable}{}
\begin{table*}[ht!]{1}
\begin{center}
\begin{tabular}{|c|c|}
\hline
\textbf{Options} & \textbf{Explanation} \\
\hline
\text{$\vec{A}$ is similar to a diagonal matrix over $\mathbb{C}$}
&\\Using Spectral Theorem &  Every real orthogonal matrix is diagonalizable over $\mathbb{C}$\\
& $\vec{A}$ is orthogonal from above.\\
& Since,$x,y \in \mathbb{R}$ .So,$\vec{A}$ is a real orthogonal matrix.\\
&\\
$\vec{A}=\myvec{x&y\\-y&x}$&$det(\vec{A}-\lambda\vec{I}))=0$\\
&$(x-\lambda)^2+y^2=0$\\
&$\lambda_1=x-iy \qquad \lambda_2=x+iy$\\
& For two eigen values $\lambda_1,\lambda_2$ let heir corresponding eigen vectors be\\& $\vec{V_1},\vec{V_2}$\\
Finding $\vec{V_1}$  & 
$(\vec{A}-\lambda_1\vec{I})\vec{V_1}=0$\\
&$(\vec{A}-\lambda_1\vec{I})=\myvec{iy&y\\-y&iy}$\\
&By Elementary row operations we get,\\
&$(\vec{A}-\lambda_1\vec{I})=\myvec{iy&y\\0&0}$\\
&$\vec{V_1}=\myvec{i\\1}$\\
Finding $\vec{V_2}$  & 
$(\vec{A}-\lambda_2\vec{I})\vec{V_2}=0$\\
&$(\vec{A}-\lambda_2\vec{I})=\myvec{-iy&y\\-y&-iy}$\\
&By Elementary row operations we get,\\
&$(\vec{A}-\lambda_2\vec{I})=\myvec{-iy&y\\0&0}$\\
&$\vec{V_2}=\myvec{-i\\1}$\\
$\vec{A}=\vec{P}\vec{D}\vec{P^{-1}}$&$\vec{P}$ is a matrix containing eigen vectors of $\vec{A}$\\&,$\vec{D}$ is the diagonal matrix where diagonals are the eigen values of $\vec{A}$\\
&$\vec{P^{-1}}=\frac{1}{2i}\myvec{1&i\\-1&i}$\\
&$\vec{A}=\frac{1}{2i}\myvec{i&-i\\1&1}\myvec{x-iy&0\\0&x+iy}\myvec{1&i\\-1&i}$\\
&\\
&Hence,$\vec{A}$ is similar to a diagonal matrix over $\mathbb{C}$\\
&Hence,correct.\\
\hline
\end{tabular}
\caption{Finding Correct Option}
\label{table1}
\end{center}
\end{table*}

 
\end{document}
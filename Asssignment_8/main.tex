\documentclass[journal,12pt,twocolumn]{IEEEtran}
%
\usepackage{setspace}
\usepackage{gensymb}
\usepackage{siunitx}
\usepackage{tkz-euclide} 
\usepackage{textcomp}
\usepackage{standalone}
\usetikzlibrary{calc}
\newcommand\hmmax{0}
\newcommand\bmmax{0}

%\doublespacing
\singlespacing

%\usepackage{graphicx}
%\usepackage{amssymb}
%\usepackage{relsize}
\usepackage[cmex10]{amsmath}
%\usepackage{amsthm}
%\interdisplaylinepenalty=2500
%\savesymbol{iint}
%\usepackage{txfonts}
%\restoresymbol{TXF}{iint}
%\usepackage{wasysym}
\usepackage{amsthm}
%\usepackage{iithtlc}
\usepackage{mathrsfs}
\usepackage{txfonts}
\usepackage{stfloats}
\usepackage{bm}
\usepackage{cite}
\usepackage{cases}
\usepackage{subfig}
%\usepackage{xtab}
\usepackage{longtable}
\usepackage{multirow}
%\usepackage{algorithm}
%\usepackage{algpseudocode}
\usepackage{enumitem}
\usepackage{mathtools}
\usepackage{steinmetz}
\usepackage{tikz}
\usepackage{circuitikz}
\usepackage{verbatim}
\usepackage{tfrupee}
\usepackage[breaklinks=true]{hyperref}
%\usepackage{stmaryrd}
\usepackage{tkz-euclide} % loads  TikZ and tkz-base
%\usetkzobj{all}
\usetikzlibrary{calc,math}
\usepackage{listings}
    \usepackage{color}                                            %%
    \usepackage{array}                                            %%
    \usepackage{longtable}                                        %%
    \usepackage{calc}                                             %%
    \usepackage{multirow}                                         %%
    \usepackage{hhline}                                           %%
    \usepackage{ifthen}                                           %%
  %optionally (for landscape tables embedded in another document): %%
    \usepackage{lscape}     
\usepackage{multicol}
\usepackage{chngcntr}
\usepackage{amsmath}
\usepackage{cleveref}
%\usepackage{enumerate}

%\usepackage{wasysym}
%\newcounter{MYtempeqncnt}
\DeclareMathOperator*{\Res}{Res}
%\renewcommand{\baselinestretch}{2}
\renewcommand\thesection{\arabic{section}}
\renewcommand\thesubsection{\thesection.\arabic{subsection}}
\renewcommand\thesubsubsection{\thesubsection.\arabic{subsubsection}}

\renewcommand\thesectiondis{\arabic{section}}
\renewcommand\thesubsectiondis{\thesectiondis.\arabic{subsection}}
\renewcommand\thesubsubsectiondis{\thesubsectiondis.\arabic{subsubsection}}

% correct bad hyphenation here
\hyphenation{op-tical net-works semi-conduc-tor}
\def\inputGnumericTable{}                                 %%

\lstset{
%language=C,
frame=single, 
breaklines=true,
columns=fullflexible
}
%\lstset{
%language=tex,
%frame=single, 
%breaklines=true
%}
\usepackage{graphicx}
\usepackage{pgfplots}

\begin{document}


\newtheorem{theorem}{Theorem}[section]
\newtheorem{problem}{Problem}
\newtheorem{proposition}{Proposition}[section]
\newtheorem{lemma}{Lemma}[section]
\newtheorem{corollary}[theorem]{Corollary}
\newtheorem{example}{Example}[section]
\newtheorem{definition}[problem]{Definition}
%\newtheorem{thm}{Theorem}[section] 
%\newtheorem{defn}[thm]{Definition}
%\newtheorem{algorithm}{Algorithm}[section]
%\newtheorem{cor}{Corollary}
\newcommand{\BEQA}{\begin{eqnarray}}
\newcommand{\EEQA}{\end{eqnarray}}
\newcommand{\define}{\stackrel{\triangle}{=}}
\bibliographystyle{IEEEtran}
%\bibliographystyle{ieeetr}
\providecommand{\mbf}{\mathbf}
\providecommand{\abs}[1]{\ensuremath{\left\vert#1\right\vert}}
\providecommand{\norm}[1]{\ensuremath{\left\lVert#1\right\rVert}}
\providecommand{\mean}[1]{\ensuremath{E\left[ #1 \right]}}
\providecommand{\pr}[1]{\ensuremath{\Pr\left(#1\right)}}
\providecommand{\qfunc}[1]{\ensuremath{Q\left(#1\right)}}
\providecommand{\sbrak}[1]{\ensuremath{{}\left[#1\right]}}
\providecommand{\lsbrak}[1]{\ensuremath{{}\left[#1\right.}}
\providecommand{\rsbrak}[1]{\ensuremath{{}\left.#1\right]}}
\providecommand{\brak}[1]{\ensuremath{\left(#1\right)}}
\providecommand{\lbrak}[1]{\ensuremath{\left(#1\right.}}
\providecommand{\rbrak}[1]{\ensuremath{\left.#1\right)}}
\providecommand{\cbrak}[1]{\ensuremath{\left\{#1\right\}}}
\providecommand{\lcbrak}[1]{\ensuremath{\left\{#1\right.}}
\providecommand{\rcbrak}[1]{\ensuremath{\left.#1\right\}}}
\theoremstyle{remark}
\newtheorem{rem}{Remark}
\newcommand{\sgn}{\mathop{\mathrm{sgn}}}
\providecommand{\res}[1]{\Res\displaylimits_{#1}} 
%\providecommand{\norm}[1]{\lVert#1\rVert}
\providecommand{\mtx}[1]{\mathbf{#1}}
\providecommand{\fourier}{\overset{\mathcal{F}}{ \rightleftharpoons}}
%\providecommand{\hilbert}{\overset{\mathcal{H}}{ \rightleftharpoons}}
\providecommand{\system}{\overset{\mathcal{H}}{ \longleftrightarrow}}
	%\newcommand{\solution}[2]{\textbf{Solution:}{#1}}
\newcommand{\solution}{\noindent \textbf{Solution: }}
\newcommand{\cosec}{\,\text{cosec}\,}
\providecommand{\dec}[2]{\ensuremath{\overset{#1}{\underset{#2}{\gtrless}}}}
\newcommand{\myvec}[1]{\ensuremath{\begin{pmatrix}#1\end{pmatrix}}}
\newcommand{\mydet}[1]{\ensuremath{\begin{vmatrix}#1\end{vmatrix}}}
%\numberwithin{equation}{section}
\numberwithin{equation}{subsection}
%\numberwithin{problem}{section}
%\numberwithin{definition}{section}
\makeatletter
\@addtoreset{figure}{problem}
\makeatother
\let\StandardTheFigure\thefigure
\let\vec\mathbf
%\renewcommand{\thefigure}{\theproblem.\arabic{figure}}
\renewcommand{\thefigure}{\theproblem}
%\setlist[enumerate,1]{before=\renewcommand\theequation{\theenumi.\arabic{equation}}
%\counterwithin{equation}{enumi}
%\renewcommand{\theequation}{\arabic{subsection}.\arabic{equation}}
\def\putbox#1#2#3{\makebox[0in][l]{\makebox[#1][l]{}\raisebox{\baselineskip}[0in][0in]{\raisebox{#2}[0in][0in]{#3}}}}
     \def\rightbox#1{\makebox[0in][r]{#1}}
     \def\centbox#1{\makebox[0in]{#1}}
     \def\topbox#1{\raisebox{-\baselineskip}[0in][0in]{#1}}
     
 \vspace{3cm}
 \title{Assignment 8}
 \author{Matish Singh Tanwar}
 \maketitle
 \newpage
 \bigskip
 %\renewcommand{\thefigure}{\theenumi}
 \renewcommand{\thetable}{\theenumi}
\vspace{1.0cm}
\begin{abstract}
This document contains a solution to find the foot of a perpendicular from a point on the plane using Singular Value Decomposition (SVD).
\end{abstract}
\vspace{0.5cm}
%
Download all python codes from 
\begin{lstlisting}
https://github.com/Matish007/Matrix-Theory-EE5609-/tree/master/Assignment_8/Codes
\end{lstlisting}
%
and latex-tikz codes from 
\begin{lstlisting}
https://github.com/Matish007/Matrix-Theory-EE5609-/tree/master/Assignment_8
\end{lstlisting}
%
%
\vspace{0.5mm}
\section{Problem}
Find the foot of the perpendicular from $\myvec{1\\0\\2}$ on the plane $\myvec{2&-3&1}\vec{x}=0$

\section{SOLUTION}
Let orthogonal vectors be $\vec{m_1}$ and $\vec{m_2}$ to the given normal vector $\vec{n}$. Let, $\vec{m}$ = $\myvec{a\\b\\c}$, then
\begin{align}
\vec{m^T}\vec{n} = 0\\
\myvec{a&b&c}\myvec{2\\-3\\1} = 0\\
\implies 2a-3b+c = 0
\end{align}
Let a=1 and b=0 we get,
\begin{align}
\vec{m_1} = \myvec{1\\0\\-2} \label{eq:1}
\end{align}
Let a=0 and b=1 we get,
\begin{align}
\vec{m_2} = \myvec{0\\1\\3} \label{eq:2}
\end{align}
From \eqref{eq:1} and \eqref{eq:2},
\begin{align}
\vec{M}= \myvec{1&0\\0&1\\-2&3} \label{eq:3}
\end{align}
Now solving the equation
\begin{align}
\vec{M}\vec{x} = \vec{b}\label{eq:4}
\end{align}
Substituting the given point and \eqref{eq:3} in \eqref{eq:4}
\begin{align}
\myvec{1&0\\0&1\\-2&3}\vec{x}=\myvec{1\\0\\2}\label{eq:5}
\end{align}
Using the Singular value decomposition to solve \eqref{eq:5} as follows,
\begin{align}
\vec{M}=\vec{U}\vec{\Sigma}\vec{V}^T\label{eq:6}
\end{align}
Where the columns of $\vec{V}$ are the eigen vectors of $\vec{M}^T\vec{M}$ ,the columns of $\vec{U}$ are the eigen vectors of $\vec{M}\vec{M}^T$ and $\vec{\Sigma}$ is diagonal matrix of singular value of eigenvalues of $\vec{M}^T\vec{M}$.
\begin{align}
\vec{M}^T\vec{M}=\myvec{5&-6\\-6&10}\label{eq:7}\\
\vec{M}\vec{M}^T =\myvec{1&0&-2\\0&1&3\\-2&3&13}
\end{align}
Substituting \eqref{eq:6} in \eqref{eq:4}
\begin{align}
\vec{U}\vec{\Sigma}\vec{V}^T\vec{x} = \vec{b}\\
\vec{x} = \vec{V}\vec{\Sigma^{-1}}\vec{U^T}\vec{b}\label{eq:8}
\end{align}
where $\vec{\Sigma^{-1}}$ is Moore-Penrose Pseudo-Inverse of $\vec{\Sigma}$.\\ Now finding the eigen values of $\vec{M}\vec{M}^T$
\begin{align}
\mydet{\vec{M}\vec{M}^T - \lambda\vec{I}} = 0
\end{align}
\begin{align}
\mydet{1-\lambda&0&-2\\0&1-\lambda&3\\-2&3&13-\lambda}=0\\
\implies \lambda^3-15\lambda^2+14\lambda=0
\end{align}
Hence eigen values of $\vec{M}\vec{M}^T$,
\begin{align}
\lambda_1 = 1 \quad \lambda_2 =14 \quad \lambda_3=0
\end{align}
Therefore eigen vectors of $\vec{M}\vec{M}^T$,
\begin{align}
\vec{u_1}=\myvec{\frac{3}{2}\\1\\0} \quad
\vec{u_2}=\myvec{\frac{-2}{13}\\\frac{3}{13}\\1} \quad
\vec{u_3}=\myvec{2\\-3\\1}
\end{align}
Normalizing the eigen vectors,
\begin{align}
\vec{u_1}=\myvec{\frac{3}{\sqrt{13}}\\\frac{2}{\sqrt{13}}\\0}\quad
\vec{u_2}=\myvec{\frac{-2}{\sqrt{182}}\\\frac{3}{\sqrt{182}}\\\frac{13}{\sqrt{182}}}\quad
\vec{u_3}=\myvec{\frac{2}{\sqrt{14}}\\\frac{-3}{\sqrt{14}}\\\frac{1}{\sqrt{14}}}
\end{align}
Hence from the above we get,
\begin{align}
\vec{U}=\myvec{\frac{3}{\sqrt{13}}&\frac{-2}{\sqrt{182}}&\frac{2}{\sqrt{14}}\\\frac{2}{\sqrt{13}}&\frac{3}{\sqrt{182}}&\frac{-3}{\sqrt{14}}\\0&\frac{13}{\sqrt{182}}&\frac{1}{\sqrt{14}}}\label{eq:9}
\end{align}
By computing the singular values from eigen values $\lambda_1, \lambda_2, \lambda_3$ we get $\vec{\Sigma}$ as,
\begin{align}
\vec{\Sigma} = \myvec{1&0\\0&14\\0&0}
\end{align}
Now calculating eigen values of $\vec{M}^T\vec{M}$
\begin{align}
\mydet{\vec{M}^T\vec{M}-\lambda I}=0\\
\mydet{5-\lambda&-6\\-6&10-\lambda}=0\\
\implies\lambda^2-15\lambda+14 =0
\end{align}
hence the eigen values of $\vec{M}^T\vec{M}$
\begin{align}
\lambda_1 = 1 \quad \lambda_2 =14
\end{align}
Therefore eigen vectors $\vec{M}^T\vec{M}$ are,
\begin{align}
\vec{v_1}=\myvec{\frac{3}{2}\\1}\quad \vec{v_2}=\myvec{\frac{-2}{3}\\1}
\end{align}
Normalizing the eigen vectors,
\begin{align}
\vec{v_1}=\myvec{\frac{3}{\sqrt{13}}\\\frac{2}{\sqrt{13}}} \quad
\vec{v_2}=\myvec{\frac{-2}{\sqrt{13}}\\\frac{3}{\sqrt{13}}}
\end{align}
Hence $\vec{V}$ is given as,
\begin{align}
\vec{V}= \myvec{\frac{3}{\sqrt{13}}&\frac{-2}{\sqrt{13}}\\\frac{2}{\sqrt{13}}&\frac{3}{\sqrt{13}}} \label{eq:10}
\end{align}
Moore Pseudo inverse of $\Sigma$ is,
\begin{align}
\vec{\Sigma^{-1}} = \myvec{1&0&0\\0&\frac{1}{\sqrt{14}}&0} \label{eq:11}
\end{align}
Substituting \eqref{eq:9}, \eqref{eq:10} and \eqref{eq:11} in \eqref{eq:8},
\begin{align}
\vec{U}^T\vec{b}=\myvec{\frac{3}{\sqrt{13}}&\frac{2}{\sqrt{13}}&0\\\frac{-2}{\sqrt{182}}&\frac{3}{\sqrt{182}}&\frac{13}{\sqrt{182}}\\\frac{2}{\sqrt{14}}&\frac{-3}{\sqrt{14}}&\frac{1}{\sqrt{14}}}\myvec{1\\0\\2} = \myvec{\frac{3}{\sqrt{13}}\\\frac{24}{\sqrt{182}}\\\frac{4}{\sqrt{14}}}\\
\vec{\Sigma^{-1}}\vec{U}^T\vec{b}=\myvec{1&0&0\\0&\frac{1}{\sqrt{14}}&0}\myvec{\frac{3}{\sqrt{13}}\\\frac{24}{\sqrt{182}}\\\frac{4}{\sqrt{14}}}=\myvec{\frac{3}{\sqrt{13}}\\\frac{12}{7\sqrt{13}}}\\
\vec{V}\vec{\Sigma^{-1}}\vec{U}^T\vec{b}=\myvec{\frac{3}{\sqrt{13}}&\frac{-2}{\sqrt{13}}\\\frac{2}{\sqrt{13}}&\frac{3}{\sqrt{13}}}\myvec{\frac{3}{\sqrt{13}}\\\frac{12}{7\sqrt{13}}}=\myvec{\frac{3}{7}\\\frac{6}{7}}\\
\implies\vec{x}=\myvec{\frac{3}{7}\\\frac{6}{7}}\label{eq:12}
\end{align}
Now verifying \eqref{eq:12} using \eqref{eq:4}
\begin{align}
\vec{M}\vec{x}=\vec{b}
\implies\vec{M}^T\vec{M}\vec{x} = \vec{M}^T\vec{b}\label{eq:13}
\end{align}
Substituting \eqref{eq:3}, \eqref{eq:7} and given point in \eqref{eq:13}
\begin{align}
\myvec{5&-6\\-6&10}\vec{x} = \myvec{-3\\6}\\
\end{align}
Solving the augmented matrix.
\begin{align}
\myvec{5&-6&-3\\-6&10&6}\xleftrightarrow{R_1=\frac{R_1}{5}}\myvec{1&\frac{-6}{5}&\frac{-3}{5}\\-6&10&6}\\
\xleftrightarrow{R_2=R_2+6R_1}\myvec{1&\frac{-6}{5}&\frac{-3}{5}\\0&\frac{14}{5}&\frac{12}{5}}\\
\xleftrightarrow{R_2=\frac{5R_2}{14}}\myvec{1&\frac{-6}{5}&\frac{-3}{5}\\0&1&\frac{6}{7}}\\
\xleftrightarrow{R_1=R_1+\frac{6R_2}{5}}\myvec{1&0&\frac{3}{7}\\0&1&\frac{6}{7}}\label{eq:14}
\end{align}
From \eqref{eq:14} we get,
\begin{align}
\vec{x}=\myvec{\frac{3}{7}\\\frac{6}{7}}\label{eq:15}
\end{align}
Hence from \eqref{eq:12} and \eqref{eq:15} the $\vec{x}$ is verified

\end{document}

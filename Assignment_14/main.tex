\documentclass[journal,12pt,twocolumn]{IEEEtran}
%
\usepackage{setspace}
\usepackage{gensymb}
\usepackage{siunitx}
\usepackage{tkz-euclide} 
\usepackage{textcomp}
\usepackage{standalone}
\usetikzlibrary{calc}
\newcommand\hmmax{0}
\newcommand\bmmax{0}

%\doublespacing
\singlespacing

%\usepackage{graphicx}
%\usepackage{amssymb}
%\usepackage{relsize}
\usepackage[cmex10]{amsmath}
%\usepackage{amsthm}
%\interdisplaylinepenalty=2500
%\savesymbol{iint}
%\usepackage{txfonts}
%\restoresymbol{TXF}{iint}
%\usepackage{wasysym}
\usepackage{amsthm}
%\usepackage{iithtlc}
\usepackage{mathrsfs}
\usepackage{txfonts}
\usepackage{stfloats}
\usepackage{bm}
\usepackage{cite}
\usepackage{cases}
\usepackage{subfig}
%\usepackage{xtab}
\usepackage{longtable}
\usepackage{multirow}
%\usepackage{algorithm}
%\usepackage{algpseudocode}
\usepackage{enumitem}
\usepackage{mathtools}
\usepackage{steinmetz}
\usepackage{tikz}
\usepackage{circuitikz}
\usepackage{verbatim}
\usepackage{tfrupee}
\usepackage[breaklinks=true]{hyperref}
%\usepackage{stmaryrd}
\usepackage{tkz-euclide} % loads  TikZ and tkz-base
%\usetkzobj{all}
\usetikzlibrary{calc,math}
\usepackage{listings}
    \usepackage{color}                                            %%
    \usepackage{array}                                            %%
    \usepackage{longtable}                                        %%
    \usepackage{calc}                                             %%
    \usepackage{multirow}                                         %%
    \usepackage{hhline}                                           %%
    \usepackage{ifthen}                                           %%
  %optionally (for landscape tables embedded in another document): %%
    \usepackage{lscape}     
\usepackage{multicol}
\usepackage{chngcntr}
\usepackage{amsmath}
\usepackage{cleveref}
%\usepackage{enumerate}

%\usepackage{wasysym}
%\newcounter{MYtempeqncnt}
\DeclareMathOperator*{\Res}{Res}
%\renewcommand{\baselinestretch}{2}
\renewcommand\thesection{\arabic{section}}
\renewcommand\thesubsection{\thesection.\arabic{subsection}}
\renewcommand\thesubsubsection{\thesubsection.\arabic{subsubsection}}

\renewcommand\thesectiondis{\arabic{section}}
\renewcommand\thesubsectiondis{\thesectiondis.\arabic{subsection}}
\renewcommand\thesubsubsectiondis{\thesubsectiondis.\arabic{subsubsection}}

% correct bad hyphenation here
\hyphenation{op-tical net-works semi-conduc-tor}
\def\inputGnumericTable{}                                 %%

\lstset{
%language=C,
frame=single, 
breaklines=true,
columns=fullflexible
}
%\lstset{
%language=tex,
%frame=single, 
%breaklines=true
%}
\usepackage{graphicx}
\usepackage{pgfplots}

\begin{document}


\newtheorem{theorem}{Theorem}[section]
\newtheorem{problem}{Problem}
\newtheorem{proposition}{Proposition}[section]
\newtheorem{lemma}{Lemma}[section]
\newtheorem{corollary}[theorem]{Corollary}
\newtheorem{example}{Example}[section]
\newtheorem{definition}[problem]{Definition}
%\newtheorem{thm}{Theorem}[section] 
%\newtheorem{defn}[thm]{Definition}
%\newtheorem{algorithm}{Algorithm}[section]
%\newtheorem{cor}{Corollary}
\newcommand{\BEQA}{\begin{eqnarray}}
\newcommand{\EEQA}{\end{eqnarray}}
\newcommand{\define}{\stackrel{\triangle}{=}}
\bibliographystyle{IEEEtran}
%\bibliographystyle{ieeetr}
\providecommand{\mbf}{\mathbf}
\providecommand{\abs}[1]{\ensuremath{\left\vert#1\right\vert}}
\providecommand{\norm}[1]{\ensuremath{\left\lVert#1\right\rVert}}
\providecommand{\mean}[1]{\ensuremath{E\left[ #1 \right]}}
\providecommand{\pr}[1]{\ensuremath{\Pr\left(#1\right)}}
\providecommand{\qfunc}[1]{\ensuremath{Q\left(#1\right)}}
\providecommand{\sbrak}[1]{\ensuremath{{}\left[#1\right]}}
\providecommand{\lsbrak}[1]{\ensuremath{{}\left[#1\right.}}
\providecommand{\rsbrak}[1]{\ensuremath{{}\left.#1\right]}}
\providecommand{\brak}[1]{\ensuremath{\left(#1\right)}}
\providecommand{\lbrak}[1]{\ensuremath{\left(#1\right.}}
\providecommand{\rbrak}[1]{\ensuremath{\left.#1\right)}}
\providecommand{\cbrak}[1]{\ensuremath{\left\{#1\right\}}}
\providecommand{\lcbrak}[1]{\ensuremath{\left\{#1\right.}}
\providecommand{\rcbrak}[1]{\ensuremath{\left.#1\right\}}}
\theoremstyle{remark}
\newtheorem{rem}{Remark}
\newcommand{\sgn}{\mathop{\mathrm{sgn}}}
\providecommand{\res}[1]{\Res\displaylimits_{#1}} 
%\providecommand{\norm}[1]{\lVert#1\rVert}
\providecommand{\mtx}[1]{\mathbf{#1}}
\providecommand{\fourier}{\overset{\mathcal{F}}{ \rightleftharpoons}}
%\providecommand{\hilbert}{\overset{\mathcal{H}}{ \rightleftharpoons}}
\providecommand{\system}{\overset{\mathcal{H}}{ \longleftrightarrow}}
	%\newcommand{\solution}[2]{\textbf{Solution:}{#1}}
\newcommand{\solution}{\noindent \textbf{Solution: }}
\newcommand{\cosec}{\,\text{cosec}\,}
\providecommand{\dec}[2]{\ensuremath{\overset{#1}{\underset{#2}{\gtrless}}}}
\newcommand{\myvec}[1]{\ensuremath{\begin{pmatrix}#1\end{pmatrix}}}
\newcommand{\mydet}[1]{\ensuremath{\begin{vmatrix}#1\end{vmatrix}}}
%\numberwithin{equation}{section}
\numberwithin{equation}{subsection}
%\numberwithin{problem}{section}
%\numberwithin{definition}{section}
\makeatletter
\@addtoreset{figure}{problem}
\makeatother
\let\StandardTheFigure\thefigure
\let\vec\mathbf
%\renewcommand{\thefigure}{\theproblem.\arabic{figure}}
\renewcommand{\thefigure}{\theproblem}
%\setlist[enumerate,1]{before=\renewcommand\theequation{\theenumi.\arabic{equation}}
%\counterwithin{equation}{enumi}
%\renewcommand{\theequation}{\arabic{subsection}.\arabic{equation}}
\def\putbox#1#2#3{\makebox[0in][l]{\makebox[#1][l]{}\raisebox{\baselineskip}[0in][0in]{\raisebox{#2}[0in][0in]{#3}}}}
     \def\rightbox#1{\makebox[0in][r]{#1}}
     \def\centbox#1{\makebox[0in]{#1}}
     \def\topbox#1{\raisebox{-\baselineskip}[0in][0in]{#1}}
     
 \vspace{3cm}
 \title{Assignment 14}
 \author{Matish Singh Tanwar}
 \maketitle
 \newpage
 \bigskip
 %\renewcommand{\thefigure}{\theenumi}
 \renewcommand{\thetable}{\theenumi}
\vspace{1.0cm}
\begin{abstract}
This document solves a problem of Linear Algebra.
\end{abstract}
\vspace{0.5cm}
%
Download all latex-tikz codes from 
\begin{lstlisting}
https://github.com/Matish007/Matrix-Theory-EE5609-/tree/master/Assignment_14
\end{lstlisting}
%
%
\vspace{0.5mm}
\section{Problem}
Let $B: \mathbb{R} \times \mathbb{R} \xrightarrow[]{} \mathbb{R}$ be the function $B(a,b) = ab$. Which of the following is true-
\begin{enumerate}
\item{$B$ is a linear transformation}
\item{$B$ is a positive definite bilinear form}
\item{$B$ is symmetric but not positive definite}
\item{$B$ neither linear nor bilinear}
\end{enumerate}
\section{SOLUTION}
 Let
 \begin{align}
    \vec{x} = \myvec{x&y}^T 
 \end{align}
 Then
 \begin{align}
     B(x,y) = \vec{x}^T\frac{\vec{R}}{2}\vec{x}\label{1} 
 \end{align}
 where $\vec{R}$ is the reflection matrix defined as:-
 \begin{align}
  \myvec{0 & 1 \\ 1 & 0}
 \end{align}
 \eqref{1} represent Quadratic form of B(x,y)
\renewcommand{\thetable}{1}
\begin{table*}[ht!]
\begin{center}
\begin{tabular}{|c|c|}
\hline
\textbf{Options} & \textbf{Explanation} \\
\hline
\text{$B$ is a linear transformation} & 
Let the transformation be $B: \mathbb{R} \times \mathbb{R} \xrightarrow[]{} \mathbb{R}$ such that, 
\\& $B(\vec{x}) = xy$ where $\vec{x} = \myvec{x\\y}$
\\& Now $B(\vec{e}) = ab$ where $\vec{e} = \myvec{a\\b}$ 
\\& Hence, $B(c\vec{e}) = c^2B(\vec{e})$
\\& Hence $B$ is not a linear transformation.\\
& Hence incorrect.
\\
\hline
\text{$B$ is a positive definite bilinear form} & 
$f: \mathbb{V} \times \mathbb{V} \xrightarrow[]{} \mathbb{F}$ where $\mathbb{V}$ is a vector space and $\mathbb{F}$ is a field\\
Bilinear Form&$f$ is a bilinear if the following holds true - 
\\& If one variable is fixed then other should be linear  
\\&Let's say $x$ is fixed,$x$=c
\\&$\eqref{1}$ becomes $B(x,y)=cy$,$y$ is linear
\\&Let's say $y$ is fixed,$y$=c
\\&$\eqref{1}$ becomes $B(x,y)=cx$,$x$ is linear
\\& Hence $B$ is a bilinear form.
\\Symmetric & Again a bilinear form $f$ is symmetric if $f(\alpha,\beta) = f(\beta,\alpha)$
\\& Here, $B(a,b) = ab$,from $\eqref{1}$
\\& $B(b,a) = ba$,from $\eqref{1}$
\\& $ba=ab$,Hence $B$ is symmetric.
\\Positive Definite& A symmetric bilinear $f$ is positive definite if
\\& $f(\alpha,\alpha) >0$ $\forall \alpha \ne 0$
\\& Here, $B(a,a) = a^2$ from $\eqref{1}$
\\& $a^2 > 0$ $\forall a\ne0$
\\& \textbf{Conclusion:} $B$ is symmetric and positive definite bilinear form.\\
& Hence Correct.
\\
\hline
\text{$B$ is symmetric but not positive definite}
& From previous proof it is obvious that
\\& $B$ is both symmetric as well as positive definite\\
& Hence incorrect
\\
\hline
\text{$B$ neither linear nor bilinear}
& From previous proofs it is obvious that
\\& $B$ is bilinear.\\
& Hence incorrect.
\\
\hline
\text{Result}
& $B$ is symmetric and positive definite bilinear form
\\
\hline
\end{tabular}
\caption{Finding Correct Option}
\label{table1}
\end{center}
\end{table*}

 
\end{document}
\documentclass[journal,12pt,twocolumn]{IEEEtran}

\usepackage{setspace}
\usepackage{gensymb}

\singlespacing

\usepackage[cmex10]{amsmath}
\usepackage{amsthm}
\usepackage{mathrsfs}
\usepackage{txfonts}
\usepackage{stfloats}
\usepackage{bm}
\usepackage{cite}
\usepackage{cases}
\usepackage{subfig}
\usepackage{longtable}
\usepackage{multirow}
\usepackage{mathtools}
\usepackage{steinmetz}
\usepackage{tikz}
\usepackage{circuitikz}
\usepackage{verbatim}
\usepackage{tfrupee}
\usepackage[breaklinks=true]{hyperref}
\usepackage{tkz-euclide} % loads  TikZ and tkz-base
%\usetkzobj{all}
\usetikzlibrary{calc,math}
\usepackage{listings}
    \usepackage{color}                                            %%
    \usepackage{array}                                            %%
    \usepackage{longtable}                                        %%
    \usepackage{calc}                                             %%
    \usepackage{multirow}                                         %%
    \usepackage{hhline}                                           %%
    \usepackage{ifthen}                                           %%
  %optionally (for landscape tables embedded in another document): %%
    \usepackage{lscape}     
\usepackage{multicol}
\usepackage{chngcntr}
\DeclareMathOperator*{\Res}{Res}
\renewcommand\thesection{\arabic{section}}
\renewcommand\thesubsection{\thesection.\arabic{subsection}}
\renewcommand\thesubsubsection{\thesubsection.\arabic{subsubsection}}

\renewcommand\thesectiondis{\arabic{section}}
\renewcommand\thesubsectiondis{\thesectiondis.\arabic{subsection}}
\renewcommand\thesubsubsectiondis{\thesubsectiondis.\arabic{subsubsection}}

\newcommand{\bignorm}[1]{\Bigl \| #1 \Bigr \| #1}
\newcommand{\norm}[1]{\| #1 \|}
% correct bad hyphenation here
\hyphenation{op-tical net-works semi-conduc-tor}
\def\inputGnumericTable{}                                 %%

\lstset{
frame=single, 
breaklines=true,
columns=fullflexible
}


\begin{document}
\begin{center}
\huge Assignment 1\\

\large Matish Singh Tanwar\\
\large AI20MTECH11005\\
\end{center}
\vspace{1.0cm}
\begin{abstract}
This document finds a unit vector parallel to a given vector
\end{abstract}
\vspace{0.5cm}
Download all python codes from 
\begin{lstlisting}
https://github.com/Matish007/Matrix-Theory-EE5609-/tree/master/codes
\end{lstlisting}
%
and latex-tikz codes from 
\begin{lstlisting}
https://github.com/Matish007/Matrix-Theory-EE5609-
\end{lstlisting}
%
\vspace{0.5mm}
\section{Problem}
Find a unit vector parallel to $\bm{2a}-\bm{b}+\bm{3c}$\\
$\bm{a}$= $\begin{pmatrix}1 \\1 \\1\end{pmatrix}$,$\bm{b}$=$\begin{pmatrix}2 \\-1 \\3\end{pmatrix}$,$\bm{c}$=$\begin{pmatrix}1 \\-2 \\1\end{pmatrix}$\\
\section{Explanation}
First calculate $\bm{2a}-\bm{b}+\bm{3c}$.Then divide the resultant vector with its magnitude,that will be a unit vector parallel to $\bm{2a}-\bm{b}+\bm{3c}$\\
\vspace{2mm}
\begin{align}
    \bm{d}=\bm{2a}-\bm{b}+\bm{3c}\\
    \bm{2a}=\begin{pmatrix}2\\2\\2\end{pmatrix}\\
    \bm{-b}=\begin{pmatrix}-2\\1\\-3\end{pmatrix}\\
    \bm{3c}=\begin{pmatrix}3\\-6\\3\end{pmatrix}
\end{align}
\vspace{2mm}\\
Substituting (2),(3),(4) in (1) we get\\
\begin{align}
    \bm{d}=\begin{pmatrix}3\\-3\\2\end{pmatrix}\\
    \norm{\bm{d}}=\sqrt{3^2+(-3)^2+2^2}=\sqrt{22}\\
    \bm{e}=\frac{\bm{d}}{\norm{\bm{d}}}
\end{align}
'e'is the unit vector parallel to given vector
Substituting (5),(6) in (7) we get
\begin{align}
    \boxed{\bm{e}=\frac{1}{\sqrt{22}}\begin{pmatrix}3\\-3\\2\end{pmatrix}}
\end{align}
Equation 8 gives us a unit vector $\bm{e}$ parallel to\\ $\bm{2a}-\bm{b}+\bm{3c}$  
\end{document}
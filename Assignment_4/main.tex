\documentclass[journal,12pt,twocolumn]{IEEEtran}

 \usepackage{setspace}
 \usepackage{gensymb}
 \usepackage{graphicx}

 \singlespacing


 \usepackage{amsmath}

 \usepackage{amsthm}
 \usepackage{mathrsfs}
 \usepackage{txfonts}
 \usepackage{stfloats}
 \usepackage{bm}
 \usepackage{cite}
 \usepackage{cases}
 \usepackage{subfig}
 \usepackage{longtable}
 \usepackage{multirow}
 \usepackage{commath}
 \usepackage{enumitem}
 \usepackage{mathtools}
 \usepackage{steinmetz}
 \usepackage{tikz}
 \usepackage{circuitikz}
 \usepackage{verbatim}
 \usepackage{tfrupee}
 \usepackage[breaklinks=true]{hyperref}
 \usepackage{tkz-euclide}

 \usetikzlibrary{calc,math}
 \usepackage{listings}
     \usepackage{color}                                            %%
     \usepackage{array}                                            %%
     \usepackage{longtable}                                        %%
     \usepackage{calc}                                             %%
     \usepackage{multirow}                                         %%
     \usepackage{hhline}                                           %%
     \usepackage{ifthen}                                           %%
     \usepackage{lscape}     
 \usepackage{multicol}
 \usepackage{chngcntr}

 \DeclareMathOperator*{\Res}{Res}

 \renewcommand\thesection{\arabic{section}}
 \renewcommand\thesubsection{\thesection.\arabic{subsection}}
 \renewcommand\thesubsubsection{\thesubsection.\arabic{subsubsection}}

 \renewcommand\thesectiondis{\arabic{section}}
 \renewcommand\thesubsectiondis{\thesectiondis.\arabic{subsection}}
 \renewcommand\thesubsubsectiondis{\thesubsectiondis.\arabic{subsubsection}}


 \hyphenation{op-tical net-works semi-conduc-tor}
 \def\inputGnumericTable{}                                 %%

 \lstset{
 %language=C,
 frame=single, 
 breaklines=true,
 columns=fullflexible
 }
\begin{document}


 \newtheorem{theorem}{Theorem}[section]
 \newtheorem{problem}{Problem}
 \newtheorem{proposition}{Proposition}[section]
 \newtheorem{lemma}{Lemma}[section]
 \newtheorem{corollary}[theorem]{Corollary}
 \newtheorem{example}{Example}[section]
 \newtheorem{definition}[problem]{Definition}

 \newcommand{\BEQA}{\begin{eqnarray}}
 \newcommand{\EEQA}{\end{eqnarray}}
 \newcommand{\define}{\stackrel{\triangle}{=}}
 \bibliographystyle{IEEEtran}
 \providecommand{\mbf}{\mathbf}
 \providecommand{\pr}[1]{\ensuremath{\Pr\left(#1\right)}}
 \providecommand{\qfunc}[1]{\ensuremath{Q\left(#1\right)}}
 \providecommand{\sbrak}[1]{\ensuremath{{}\left[#1\right]}}
 \providecommand{\lsbrak}[1]{\ensuremath{{}\left[#1\right.}}
 \providecommand{\rsbrak}[1]{\ensuremath{{}\left.#1\right]}}
 \providecommand{\brak}[1]{\ensuremath{\left(#1\right)}}
 \providecommand{\lbrak}[1]{\ensuremath{\left(#1\right.}}
 \providecommand{\rbrak}[1]{\ensuremath{\left.#1\right)}}
 \providecommand{\cbrak}[1]{\ensuremath{\left\{#1\right\}}}
 \providecommand{\lcbrak}[1]{\ensuremath{\left\{#1\right.}}
 \providecommand{\rcbrak}[1]{\ensuremath{\left.#1\right\}}}
 \theoremstyle{remark}
 \newtheorem{rem}{Remark}
 \newcommand{\sgn}{\mathop{\mathrm{sgn}}}
 \providecommand{\res}[1]{\Res\displaylimits_{#1}} 
 %\providecommand{\norm}[1]{\lVert#1\rVert}
 \providecommand{\mtx}[1]{\mathbf{#1}}
 \providecommand{\fourier}{\overset{\mathcal{F}}{ \rightleftharpoons}}
 %\providecommand{\hilbert}{\overset{\mathcal{H}}{ \rightleftharpoons}}
 \providecommand{\system}{\overset{\mathcal{H}}{ \longleftrightarrow}}
 	%\newcommand{\solution}[2]{\textbf{Solution:}{#1}}
 \newcommand{\solution}{\noindent \textbf{Solution: }}
 \newcommand{\cosec}{\,\text{cosec}\,}
 \providecommand{\dec}[2]{\ensuremath{\overset{#1}{\underset{#2}{\gtrless}}}}
 \newcommand{\myvec}[1]{\ensuremath{\begin{pmatrix}#1\end{pmatrix}}}
 \newcommand{\mydet}[1]{\ensuremath{\begin{vmatrix}#1\end{vmatrix}}}
 \numberwithin{equation}{subsection}
 \makeatletter
 \@addtoreset{figure}{problem}
 \makeatother
 \let\StandardTheFigure\thefigure
 \let\vec\mathbf
 \renewcommand{\thefigure}{\theproblem}
 \def\putbox#1#2#3{\makebox[0in][l]{\makebox[#1][l]{}\raisebox{\baselineskip}[0in][0in]{\raisebox{#2}[0in][0in]{#3}}}}
      \def\rightbox#1{\makebox[0in][r]{#1}}
      \def\centbox#1{\makebox[0in]{#1}}
      \def\topbox#1{\raisebox{-\baselineskip}[0in][0in]{#1}}
      \def\midbox#1{\raisebox{-0.5\baselineskip}[0in][0in]{#1}}
 \vspace{3cm}
 \title{Assignment 4}
 \author{Matish Singh Tanwar}
 \maketitle
 \newpage
 \bigskip
 %\renewcommand{\thefigure}{\theenumi}
 \renewcommand{\thetable}{\theenumi}
\vspace{1.0cm}
\begin{abstract}
This document solves question based on triangle.
\end{abstract}
\vspace{0.5cm}
%
Download all latex-tikz codes from 
\begin{lstlisting}
https://github.com/Matish007/Matrix-Theory-EE5609-/tree/master/Assignment_4
\end{lstlisting}
%
\vspace{0.5mm}
\section{Problem}
Line L is the bisector of $\angle{A}$ and B is any point on L.BP and BQ are perpendiculars from B to the arms of $\angle{A}$.Show that:-
\begin{align}
    a) &\quad \triangle APB \cong\triangle AQB\\
    b) & \quad BP=BQ
\end{align}
\begin{figure}[!htb]
	\centering
	\includegraphics[width=\columnwidth]{fig.jpg}
	\caption{\label{fig1}}
	\label{fig:1}
\end{figure}

\section{Explanation}
Given:-
\begin{align}
  \angle{BAP}=\angle{BAQ}=\alpha\label{1}\\ 
    \angle{AQB}=\angle{APB}\label{2}
\end{align}
In $\triangle ABQ$
\begin{align}
    \angle{ABQ} + \angle{AQB} + \angle{BAQ}=180\degree\label{3}
\end{align}
In $\triangle ABP$
\begin{align}
    \angle{ABP} + \angle{APB} + \angle{BAP}=180\degree\label{4}
\end{align}
Subtracting \eqref{3} and \eqref{4} and using \eqref{1} and \eqref{2} we get,
\begin{align}
\angle{ABQ}=\angle{ABP}
\end{align}
Since,line BP and BQ are perpendicular to AP and AQ respectively..So,their respective dot product will be zero.We get,
\begin{align}
    (\vec{B}-\vec{Q})^T(\vec{A}-\vec{Q})=0\label{5}\\
    (\vec{B}-\vec{P})^T(\vec{A}-\vec{P})=0\label{6}
\end{align}
We know that,$(\vec{B}-\vec{P})^T(\vec{B}-\vec{P})$ = $\norm{\vec{B}-\vec{P}}^2$\\
 Also let
 \begin{align}
 \norm{\vec{B}-\vec{A}}^2 = k^2\label{7}
 \end{align}
\begin{align}
(\vec{B}-\vec{P})^T(\vec{B}-\vec{P})=(\vec{B}-\vec{A}+\vec{A}-\vec{P})^T(\vec{B}-\vec{A}+\vec{A}-\vec{P})
\end{align}
\begin{align}
 \begin{split}
\norm{\vec{B}-\vec{P}}^2=(\vec{B}-\vec{A})^T(\vec{B}-\vec{A})\\
+(\vec{A}-\vec{P})^T(\vec{A}-\vec{P})\\
+(\vec{A}-\vec{P})^T(\vec{B}-\vec{A})\\
+(\vec{B}-\vec{A})^T(\vec{A}-\vec{P})\\
=\norm{\vec{B}-\vec{A}}^2 + \norm{\vec{A}-\vec{P}}^2\\ 
  + 2\norm{\vec{A}-\vec{P}}\norm{\vec{B}-\vec{A}}\cos\alpha\label{8}
  \end{split}
\end{align}
Substituting \eqref{7} in \eqref{8} we get,\\
And $(\vec{A}-\vec{P})^T(\vec{B}-\vec{A})$=$(\vec{B}-\vec{A})^T(\vec{A}-\vec{P})$\\
 $=$ $\norm{\vec{A}-\vec{P}}$ $\norm{\vec{B}-\vec{A}}$ $\cos\alpha$\\
\begin{align}
  \begin{split}
      \norm{\vec{B}-\vec{P}}^2=k^2+\norm{\vec{A}-\vec{P}}^2
      +2k\norm{\vec{A}-\vec{P}}\cos\alpha\label{14}
  \end{split}  
\end{align}
Similarly,we get
\begin{align}
    \norm{\vec{B}-\vec{Q}}^2=k^2+\norm{\vec{A}-\vec{Q}}^2
      +2k\norm{\vec{A}-\vec{Q}}\cos\alpha\label{15}
\end{align}
\begin{align}
    \cos\alpha = \frac{(\vec{B}-\vec{A})^T(\vec{P}-\vec{A})}{k\norm{\vec{P}-\vec{A}}} = \frac{(\vec{B}-\vec{A})^T(\vec{Q}-\vec{A})}{k\norm{\vec{Q}-\vec{A}}}\label{13}\\
    (\vec{B}-\vec{A})^T(\vec{P}-\vec{A})=(\vec{B}-\vec{P}+\vec{P}-\vec{A})^T(\vec{P}-\vec{A})\\
    \implies (\vec{B}-\vec{P})^T(\vec{P}-\vec{A}) + \norm{\vec{P}-\vec{A}}^2\label{9}\\
    (\vec{B}-\vec{A})^T(\vec{Q}-\vec{A})=(\vec{B}-\vec{Q}+\vec{Q}-\vec{A})^T(\vec{Q}-\vec{A})\\
     \implies (\vec{B}-\vec{Q})^T(\vec{Q}-\vec{A}) + \norm{\vec{Q}-\vec{A}}^2\label{10}
\end{align}
Substituting \eqref{5} and \eqref{6} in \eqref{10} and \eqref{9} respectively we get,
\begin{align}
    (\vec{B}-\vec{A})^T(\vec{P}-\vec{A})=\norm{\vec{P}-\vec{A}}^2\label{11}\\
    (\vec{B}-\vec{A})^T(\vec{Q}-\vec{A})=\norm{\vec{Q}-\vec{A}}^2\label{12}
\end{align}
Substituting \eqref{11} and \eqref{12} in \eqref{13} we get,
\begin{align}
    \cos\alpha=\norm{\vec{P}-\vec{A}}=\norm{\vec{Q}-\vec{A}}\label{16}
\end{align}
Substituting \eqref{16} in \eqref{14} and \eqref{15} we get,
\begin{align}
    \norm{\vec{B}-\vec{P}}=\norm{\vec{B}-\vec{Q}}\label{17}
\end{align}
From \eqref{17} we can say that $\vec{B}$ is equidistant from the arms of $\angle{A}$,where $\vec{P}$ and $\vec{Q}$ are the points on the arms of $\angle{A}$
Using \eqref{1},\eqref{2},\eqref{17} and by AAS(Angle Angle Side) property of congruency we can say that:-\\
\begin{align}
\triangle APB \cong \triangle AQB
\end{align}
\end{document}